\documentclass[a4paper,12pt]{memoir}

\usepackage[margin=1in,tmargin=0.6in]{geometry}
\usepackage{polski}
\usepackage[utf8]{inputenc}
\usepackage{graphicx}
\usepackage{wrapfig}
\usepackage{lipsum}

\DeclareUnicodeCharacter{00A0}{ }

\setlength{\parindent}{0pt} 
\setlength{\parskip}{7pt} 

\pagestyle{empty}

\begin{document}

\begin{minipage}[c]{.2\linewidth}
	\includegraphics[width=\linewidth]{logo_kms.jpg}
\end{minipage}
\begin{minipage}[c]{.7\linewidth}
	\begin{center}
	Koło Matematyków Studentów Uniwersytetu Jagiellońskiego\\
	im. prof. Stanisława Zaremby

	\scriptsize{
	Instytut Matematyki UJ\\
	ul. Łojasiewicza 6/1008, 30-348 Kraków\\
	http://kmsuj.im.uj.edu.pl   e-mail: kmsuj7@gmail.com
	}
	\end{center}
\end{minipage}

\hrulefill

\begin{vplace}

\begin{center}
	\Huge\textbf{Segregator Szkocki}
\end{center}

\end{vplace}
\begin{vplace}

\begin{center}
	\Large\textbf{Problem Pierwszy}
\end{center}

\end{vplace}


\begin{vplace}

Dany jest trójkąt o wierzchołkach $A,B,C$.

Niech $E,F,G$ Będą punktami na bokach odpowiednio $AB,CB,AC$ takimi, że 
$$\frac{|AE|}{|EB|} = \frac{|BF|}{|CF|} = \frac{|GC|}{|AG|}$$
$O_{1}$ nazwijmy okrąg przechodzący przez $C$, styczny do prostej $AB$ w punkcie $E$.\\
$O_{2}$ okrąg przechodzący przez $B$, styczny do $AC$ w $G$.\\
$O_{3}$ okrąg przechodzący przez $A$, styczny do prostej $BC$ w punkcie $F$.\\
Nazwijmy punkt przecięcia osi potęgowych okręgów $O_{1},O_{2},O_{3}$ punktem $x$.

Hipoteza:\\
Istnieje elipsa, która dla każdego wyboru punktów $E,F,G$, zawiera punkt $x$.

\end{vplace}

\begin{vplace}

Nagroda:\\
10 kg ziemniaków

\begin{flushright}
	\vspace{48pt}
	Marcel Windys
\end{flushright}

\end{vplace}



\end{document}
